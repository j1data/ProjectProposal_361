\documentclass[12pt]{article}
\renewcommand{\thesection}{\Roman{section}} 
\renewcommand{\thesubsection}{\thesection.\Roman{subsection}}
%\usepackage[tocindentauto]{tocstyle}
%\usetocstyle{KOMAlike} %the previous line resets it
%\usepackage{natbib}
\usepackage{biblatex}
\addbibresource[]{ref.bib}
\usepackage{url}
\usepackage[utf8]{inputenc}
\usepackage{amsmath}
\usepackage{graphicx}
\usepackage{graphviz}
\usepackage[T1]{fontenc}
\graphicspath{{images/}}
\usepackage{parskip}
\usepackage{fancyhdr}
\usepackage{hyperref}
\usepackage{parskip}
\usepackage{hologo}
\usepackage{listings}
\usepackage{titlesec, blindtext, color}
\usepackage{titling}
\usepackage{tcolorbox}
\usepackage[hmargin=1in,vmargin=1in]{geometry}
\usepackage{float}
\usepackage{tikz}
\usepackage{appendix}
\usepackage{listings} % For code importing
\usepackage{xcolor} % for setting colors
\usepackage{svg}
\usepackage{tocloft}
\renewcommand{\cftsecleader}{\cftdotfill{\cftdotsep}}

\input{arduinoLanguage.tex}

\hypersetup{
	colorlinks=true,
	linkcolor=blue,
	urlcolor=cyan,
}

\lstdefinestyle{customc}{
  belowcaptionskip=1\baselineskip,
  breaklines=true,
  frame=L,
  xleftmargin=\parindent,
  language=C,
  showstringspaces=false,
  basicstyle=\footnotesize\ttfamily,
  keywordstyle=\bfseries\color{green!40!black},
  commentstyle=\itshape\color{purple!40!black},
  identifierstyle=\color{blue},
  stringstyle=\color{orange},
 }

 \lstset{ %
  backgroundcolor=\color{white},   % choose the background color; you must add \usepackage{color} or \usepackage{xcolor}
  basicstyle=\footnotesize,        % the size of the fonts that are used for the code
  breakatwhitespace=false,         % sets if automatic breaks should only happen at whitespace
  breaklines=true,                 % sets automatic line breaking
  captionpos=b,                    % sets the caption-position to bottom
  commentstyle=\color{commentsColor}\textit,    % comment style
  deletekeywords={...},            % if you want to delete keywords from the given language
  escapeinside={\%*}{*)},          % if you want to add LaTeX within your code
  extendedchars=true,              % lets you use non-ASCII characters; for 8-bits encodings only, does not work with UTF-8
  frame=tb,	                   	   % adds a frame around the code
  keepspaces=true,                 % keeps spaces in text, useful for keeping indentation of code (possibly needs columns=flexible)
  keywordstyle=\color{keywordsColor}\bfseries,       % keyword style
  language=Python,                 % the language of the code (can be overrided per snippet)
  otherkeywords={*,...},           % if you want to add more keywords to the set
  numbers=left,                    % where to put the line-numbers; possible values are (none, left, right)
  numbersep=8pt,                   % how far the line-numbers are from the code
  numberstyle=\tiny\color{commentsColor}, % the style that is used for the line-numbers
  rulecolor=\color{black},         % if not set, the frame-color may be changed on line-breaks within not-black text (e.g. comments (green here))
  showspaces=false,                % show spaces everywhere adding particular underscores; it overrides 'showstringspaces'
  showstringspaces=false,          % underline spaces within strings only
  showtabs=false,                  % show tabs within strings adding particular underscores
  stepnumber=1,                    % the step between two line-numbers. If it's 1, each line will be numbered
  stringstyle=\color{stringColor}, % string literal style
  tabsize=2,	                   % sets default tabsize to 2 spaces
  title=\lstname,                  % show the filename of files included with \lstinputlisting; also try caption instead of title
  columns=fixed                    % Using fixed column width (for e.g. nice alignment)
}

\lstdefinestyle{customasm}{
  belowcaptionskip=1\baselineskip,
  frame=L,
  xleftmargin=\parindent,
  language=[x86masm]Assembler,
  basicstyle=\footnotesize\ttfamily,
  commentstyle=\itshape\color{purple!40!black},
}

\lstset{escapechar=@,style=customc}

%\makeatletter
%\let\thetitle\@title

%\let\thedate\@date
%\makeatother

%\pagestyle{fancy}
%\fancyhf{}
%\rhead{\theauthor}
%\lhead{\thetitle}
%\cfoot{\thepage}

\begin{document}
\title{Project Proposal}
%%%%%%%%%%%%%%%%%%%%%%%%%%%%%%%%%%%%%%%%%%%%%%%%%%%%%%%%%%%%%%%%%%%%%%%%%%%%%%%%%%%%%%%%%

\begin{titlepage}
	\centering
    \vspace*{0.5 cm}
    \includegraphics[scale = 0.11]{isu_seal.png}\\[1.0 cm]	% University Logo
    \textsc{\LARGE IOWA STATE UNIVERSITY}\\[2.0 cm]
    \textsc{\large AEROSPACE ENGINEERING DEPARTMENT}\\[0.2 cm]
    \textsc{\large Computational Techniques for Aerospace Design}\\[0.2 cm]
	\textsc{\Large AERE 361}\\[0.5 cm]				% Course Code
	\textsc{\Large Project Proposal}\\[0.2 cm]
	\textsc{\Large Team Gimli}\\[0.2 cm]
	\rule{\linewidth}{0.2 mm} \\[0.4 cm]
	%{ \huge \bfseries \thetitle}\\
	
	
	\begin{minipage}{0.8\textwidth}
		
			\begin{flushleft} 
			\emph{Team Member Names :} \\
			Schendel, Joseph\linebreak
			Pullman, Justin\linebreak
			Nandakumar, Sooraj\linebreak
			McGill, Blake\linebreak
			Lewandowski, Andrew\linebreak
			Alexander, Kasey\linebreak
			
		\end{flushleft}
	\end{minipage}\\[2 cm]
	
	\vfill
	
\end{titlepage}

%%%%%%%%%%%%%%%%%%%%%%%%%%%%%%%%%%%%%%%%%%%%%%%%%%%%%%%%%%%%%%%%%%%%%%%%%%%%%%%%%%%%%%%%%
%\maketitle
\tableofcontents
\pagebreak
%%%%%%%%%%%%%%%%%%%%%%%%%%%%%%%%%%%%%%%%%%%%%%%%%%%%%%%%%%%%%%%%%%%%%%%%%%%%%%%%%%%%%%%%%

\section{ABSTRACT}
The abstract is a summary of your proposal. In general, your abstract should have enough information so that if I was to copy and paste your abstract into a web site, people would get the general idea of what your proposal is about. It should not go into any heavy detail, just the basics of what your project is about. The who, the what, and the why. You should keep your abstract to 200-400 words. Use this to ``hook in'' your reader.

\section{INTRODUCTION}
While the abstract and introduction may seem like it is similar, remember that your abstract should have enough information to stand on its own. The introduction is really the actual start to your proposal. Here you should introduce the project, the people involved and give a short introduction to the why you are doing this. This should be 1-3 paragraphs.

\section{FEATURES}

 
The first feature of our project is a light indicator at each parking spot. We thought of posts at the front of each parking spots, akin to parking meters, that have a red or green light indicating a taken or open parking space. We moved on from this idea, however, considering material cost of having a post like this at every single parking spot in a lot. Instead, we thought of a strip at the back end of each parking spot embedded into the pavement that has the green or red light indication. As we thought about electricity cost of having a red light on all the time, we decided to have it off if the spot is taken. This system would only run during reasonable hours during the day so that it doesn't eat up electric power in the dead of night with hundreds of active green lights.

The second feature would be a display at any and all entrances to the parking lot. This would receive data from the sensors of each parking spot and display information regarding the relative availability of parking. For example, displaying number of open spots per row, so the driver can quickly see how close they can park to their destination.

The third feature would be a website that would also receive data from the parking spot sensors and display similar information to the second feature. This would allow easy access to the information so individuals can decide beforehand if they should drive over to this parking lot based on how available parking is at a given time of day. A simple statistic to be displayed could be spots open over total spots, which not only gives the nubmer of available spots, but gives a good idea of what percentage of the lot is full. A parking lot with 100 open spots out of 125 total spots is a lot different than a parking lot with 100 open spots out of 10,000 total spots.

A fourth feature we have as a backup or if we have time would be a timer system. Each spot would have a timer and display how much time is left before the car needs to be moved. This is good information to help out parking police as well as for individuals seeing how close some spots might be to being available.





\section{PROBLEM STATEMENT}
Here you will go into more detail on what problem you hope to solve or address.  You should discuss what the problem is and why it is important to solve it. In this section, you need to be clear on what the problem is, so do not think of this as a ``light'' section. It helps to define your project.

Your team needs to do some research into the problem at hand. Becuase of that, you should have around two to three references that you are pulling from. There are lots of places you can find references from including the ISU library and Google Scholar. I would also suggest looking at Adafruit's website, as you may find inspiration or looking to improve something already there. Remember to cite your sources though. If you find something online, that can often be citation.

When you create your ``ref.bib'' file, don't forget to follow the standards for a BiBTex file. Certain things like webistes requires certain keywords for it to render properly. There are lots of sources online to help with this and many places like the ISU Library and Google Scholar can also generate text that is compatible with a BiBTex file. Once you have your Bib file ready, don't forget to cite your citations in your proposal like this \cite{einstein} or this \cite{dirac}.

\section{PROBLEM SOLUTION}
\begin{figure}[!h]
	\centering
	\includegraphics[width=6in]{parkingDiagram.jpg}
	\caption{Example and Circuit Diagram}
	\label{fig:cpx}
\end{figure}
The solution that was decided on, is to place sensors in each parking spot to detect if there is a vehicle occupying it. This sensor can be a pressure, ultrasonic, or electromagnetic sensor. When a parking spot is empty a light strip at the entrance to the parking spot will turn green indicating the spot it open and available. When a vehicle drives into the spot and over the sensor, the light will turn off and a signal will be sent to the micro-controller (Adafruit Circuit Playground Express). The micro-controller will then update an LCD display positioned at the entrance of the parking lot, indicating how many open spots there are in each row. The micro-controller will also update a website with the number of spots open out of the total number of spots and a percentage of how full the parking lot is. These three functionalities will allow drivers to check if there are parking spots prior to driving over, update drivers who are looking for an open lot, and help indicate to drivers where free parking spots are. Figure \ref{fig:cpx} shows what a parking lot row would look like with our system implemented. The light strip at the base of parking spot are green when the spot it empty and off when the spot it filled. There is a billboard LCD at the entrance to the parking lot showing the breakdown of how many empty spots there are per row. On the right side of Figure \ref{fig:cpx} there is a breakdown of the wiring diagram for 3 parking spots. Each parking spot has a sensor and a light strip, which communicate with the microprocessor which updates the website and LCD billboard. 

\begin{table}[!h]
  \caption{Parts required for project}
  \label{table:parts_list}
  \begin{center}
  \begin{tabular}{|p{3in}|c|}
  
  \hline
  Part description & Qty\\
  \hline
  \hline
  Adafruit Circuit Playground Express & 1 \\
  \hline
  AAA Battery Holder & 1 \\
  \hline
  USB Cable & 1 \\
  \hline
  Ultrasonic Sensor & 3 \\
  \hline
  Neopixel Strip & 1 \\
  \hline
  LCD & 1 \\
  \hline
  \end{tabular}
  \end{center}
\end{table}
\section{CONCLUSION}
Finally, wrap up your proposal. This only needs to be one or two paragraphs, but it should conclude with what you plan to do and the why and how. Yes, this may seem repetitive, but that is intentional. Do not forget to update your references as those will appear below in a separate page.

\newpage
%\section{References}
\printbibliography[heading=subbibintoc]
%\bibliographystyle{plain}
%\bibliography{ref}

\end{document}
