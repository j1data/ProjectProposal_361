\documentclass[12pt]{article}
\renewcommand{\thesection}{\Roman{section}} 
\renewcommand{\thesubsection}{\thesection.\Roman{subsection}}
%\usepackage[tocindentauto]{tocstyle}
%\usetocstyle{KOMAlike} %the previous line resets it
%\usepackage{natbib}
\usepackage[style=numeric,backend=bibtex]{biblatex}
\addbibresource[]{ref.bib}
\usepackage{url}
\usepackage[utf8]{inputenc}
\usepackage{amsmath}
\usepackage{graphicx}
\usepackage{graphviz}
\usepackage[T1]{fontenc}
\graphicspath{{images/}}
\usepackage{parskip}
\usepackage{fancyhdr}
\usepackage{hyperref}
\usepackage{parskip}
\usepackage{hologo}
\usepackage{listings}
\usepackage{titlesec, blindtext, color}
\usepackage{titling}
\usepackage{tcolorbox}
\usepackage[hmargin=1in,vmargin=1in]{geometry}
\usepackage{float}
\usepackage{tikz}
\usepackage{appendix}
\usepackage{listings} % For code importing
\usepackage{xcolor} % for setting colors
\usepackage{svg}
\usepackage{tocloft}
\renewcommand{\cftsecleader}{\cftdotfill{\cftdotsep}}

\input{arduinoLanguage.tex}

\hypersetup{
	colorlinks=true,
	linkcolor=blue,
	urlcolor=cyan,
}

\lstdefinestyle{customc}{
  belowcaptionskip=1\baselineskip,
  breaklines=true,
  frame=L,
  xleftmargin=\parindent,
  language=C,
  showstringspaces=false,
  basicstyle=\footnotesize\ttfamily,
  keywordstyle=\bfseries\color{green!40!black},
  commentstyle=\itshape\color{purple!40!black},
  identifierstyle=\color{blue},
  stringstyle=\color{orange},
 }

 \lstset{ %
  backgroundcolor=\color{white},   % choose the background color; you must add \usepackage{color} or \usepackage{xcolor}
  basicstyle=\footnotesize,        % the size of the fonts that are used for the code
  breakatwhitespace=false,         % sets if automatic breaks should only happen at whitespace
  breaklines=true,                 % sets automatic line breaking
  captionpos=b,                    % sets the caption-position to bottom
  commentstyle=\color{commentsColor}\textit,    % comment style
  deletekeywords={...},            % if you want to delete keywords from the given language
  escapeinside={\%*}{*)},          % if you want to add LaTeX within your code
  extendedchars=true,              % lets you use non-ASCII characters; for 8-bits encodings only, does not work with UTF-8
  frame=tb,	                   	   % adds a frame around the code
  keepspaces=true,                 % keeps spaces in text, useful for keeping indentation of code (possibly needs columns=flexible)
  keywordstyle=\color{keywordsColor}\bfseries,       % keyword style
  language=Python,                 % the language of the code (can be overrided per snippet)
  otherkeywords={*,...},           % if you want to add more keywords to the set
  numbers=left,                    % where to put the line-numbers; possible values are (none, left, right)
  numbersep=8pt,                   % how far the line-numbers are from the code
  numberstyle=\tiny\color{commentsColor}, % the style that is used for the line-numbers
  rulecolor=\color{black},         % if not set, the frame-color may be changed on line-breaks within not-black text (e.g. comments (green here))
  showspaces=false,                % show spaces everywhere adding particular underscores; it overrides 'showstringspaces'
  showstringspaces=false,          % underline spaces within strings only
  showtabs=false,                  % show tabs within strings adding particular underscores
  stepnumber=1,                    % the step between two line-numbers. If it's 1, each line will be numbered
  stringstyle=\color{stringColor}, % string literal style
  tabsize=2,	                   % sets default tabsize to 2 spaces
  title=\lstname,                  % show the filename of files included with \lstinputlisting; also try caption instead of title
  columns=fixed                    % Using fixed column width (for e.g. nice alignment)
}

\lstdefinestyle{customasm}{
  belowcaptionskip=1\baselineskip,
  frame=L,
  xleftmargin=\parindent,
  language=[x86masm]Assembler,
  basicstyle=\footnotesize\ttfamily,
  commentstyle=\itshape\color{purple!40!black},
}

\lstset{escapechar=@,style=customc}

%\makeatletter
%\let\thetitle\@title

%\let\thedate\@date
%\makeatother

%\pagestyle{fancy}
%\fancyhf{}
%\rhead{\theauthor}
%\lhead{\thetitle}
%\cfoot{\thepage}

\begin{document}
\title{Project Proposal}
%%%%%%%%%%%%%%%%%%%%%%%%%%%%%%%%%%%%%%%%%%%%%%%%%%%%%%%%%%%%%%%%%%%%%%%%%%%%%%%%%%%%%%%%%

\begin{titlepage}
	\centering
    \vspace*{0.5 cm}
    \includegraphics[scale = 0.11]{isu_seal.png}\\[1.0 cm]	% University Logo
    \textsc{\LARGE IOWA STATE UNIVERSITY}\\[2.0 cm]
    \textsc{\large AEROSPACE ENGINEERING DEPARTMENT}\\[0.2 cm]
    \textsc{\large Computational Techniques for Aerospace Design}\\[0.2 cm]
	\textsc{\Large AERE 361}\\[0.5 cm]				% Course Code
	\textsc{\Large Project Proposal}\\[0.2 cm]
	\textsc{\Large Team Gimli}\\[0.2 cm]
	\rule{\linewidth}{0.2 mm} \\[0.4 cm]
	%{ \huge \bfseries \thetitle}\\
	
	
	\begin{minipage}{0.8\textwidth}
		
			\begin{flushleft} 
			\emph{Team Member Names :} \\
			Schendel, Joseph\linebreak
			Pullman, Justin\linebreak
			Nandakumar, Sooraj\linebreak
			McGill, Blake\linebreak
			Lewandowski, Andrew\linebreak
			Alexander, Kasey\linebreak
			
		\end{flushleft}
	\end{minipage}\\[2 cm]
	
	\vfill
	
\end{titlepage}

%%%%%%%%%%%%%%%%%%%%%%%%%%%%%%%%%%%%%%%%%%%%%%%%%%%%%%%%%%%%%%%%%%%%%%%%%%%%%%%%%%%%%%%%%
%\maketitle
\tableofcontents
\pagebreak
%%%%%%%%%%%%%%%%%%%%%%%%%%%%%%%%%%%%%%%%%%%%%%%%%%%%%%%%%%%%%%%%%%%%%%%%%%%%%%%%%%%%%%%%%

\section{ABSTRACT}
Finding parking spaces on ISU campus can sometimes be rather inconvenient. There have been parking solutions around the world such as available parking indicators in airports. Our project aims to implement an embedded system similar to these in the ISU armory parking lot. Our system will have multiple key features making it very appealing to both student commuters and teachers. One of these features will be a green indicator light on the parking lot ground. This light will be green when the parking spot is open and will be visible from the parking lot row. Another feature will be displaying available parking on a screen at the entrance of the parking lot. This data will also be available on a website that can be checked anywhere. We will possibly investigate implementing a timing system that can be used in place of meters. This system will sense if a car is parked in a specific spot via a pressure, ultrasonic, or electromagnetic sensor. If the sensor indicates that there is a spot open, a green light strip turn on on the ground in front of the parking spot. All open and occupied spot data will be sent to a local hub where that information can be processed and sent to display screens and website.

\section{INTRODUCTION}

The usage of embedded systems in daily life has become an everyday occurrence in the modern world. Many of these systems are in place to make life more convenient and efficient. This project is an effort to increase the efficiencies and convenience of an everyday aspect of life: finding a parking spot for a car. In an urban environment, finding an open parking spot can be a challenge involving searching through different rows in multiple lots; a problem our project aims to fix.

Our team will create a system that will track the vacancies of parking spots in a lot, to inform users on parking availability. Similar systems have been implemented in places like airports, but we decided to design a individual spot-oriented system that could be used in a parking lot on campus, such as the lot by the Armory. Users would be informed on how full a lot is and even which individual spaces are vacant.

\section{FEATURES}

 
The first feature of our project is a light indicator at each parking spot. We thought of posts at the front of each parking spots, akin to parking meters, that have a red or green light indicating a taken or open parking space. We moved on from this idea, however, considering material cost of having a post like this at every single parking spot in a lot. Instead, we thought of a strip at the back end of each parking spot embedded into the pavement that has the green or red light indication. As we thought about electricity cost of having a red light on all the time, we decided to have it off if the spot is taken. This system would only run during reasonable hours during the day so that it doesn't eat up electric power in the dead of night with hundreds of active green lights.

The second feature would be a display at any and all entrances to the parking lot. This would receive data from the sensors of each parking spot and display information regarding the relative availability of parking. For example, displaying number of open spots per row, so the driver can quickly see how close they can park to their destination.

The third feature would be a website that would also receive data from the parking spot sensors and display similar information to the second feature. This would allow easy access to the information so individuals can decide beforehand if they should drive over to this parking lot based on how available parking is at a given time of day. A simple statistic to be displayed could be spots open over total spots, which not only gives the nubmer of available spots, but gives a good idea of what percentage of the lot is full. A parking lot with 100 open spots out of 125 total spots is a lot different than a parking lot with 100 open spots out of 10,000 total spots.

A fourth feature we have as a backup or if we have time would be a timer system. Each spot would have a timer and display how much time is left before the car needs to be moved. This is good information to help out parking police as well as for individuals seeing how close some spots might be to being available.





\section{PROBLEM STATEMENT}

The problem we are trying to solve is parking on campus. It is a big headache trying to figure out where the best place to park is. Knowing how full a parking lot is, where available parking spots are, or if the best spots at a specific location are available are all problems we are trying to tackle. The solution we are going to make improves parking at different levels. It takes a tool found at airports that helps with parking and incorporates it into locations around here on campus, starting with the armory. The goal is to create a system that makes parking less of a headache by displaying information at different locations. Information such as how full the lot is and number of spots available in a specific row. Further information such as where the most open spots are, and how busy the parking lot typically is will also be explored. 

Taking inspiration from what is done at airports and implementing it at ISU is the key philosophy for this idea. At airports, wayfinding is an important concept requires consideration. "Studies have shown the importance of wayfinding and give it significant weight with respect in the determination of the overall LOS of the terminal." \cite{Harding}. Parking navigation is an example of wayfinding, and LOS simply stands for level of service. Meaning, how helpful parking directions are and parking navigation is a big thing that is considered at airports as it improves the quality of life. That is the purpose of pursuing this idea. Making parking easier for everyone. 

\section{PROBLEM SOLUTION}
\begin{figure}[!h]
	\centering
	\includegraphics[width=6in]{parkingDiagram.jpg}
	\caption{Example and Circuit Diagram}
	\label{fig:cpx}
\end{figure}
The solution that was decided on, is to place sensors in each parking spot to detect if there is a vehicle occupying it. This sensor can be a pressure, ultrasonic, or electromagnetic sensor. When a parking spot is empty a light strip at the entrance to the parking spot will turn green indicating the spot it open and available. When a vehicle drives into the spot and over the sensor, the light will turn off and a signal will be sent to the micro-controller (Adafruit Circuit Playground Express). The micro-controller will then update an LCD display positioned at the entrance of the parking lot, indicating how many open spots there are in each row. The micro-controller will also update a website with the number of spots open out of the total number of spots and a percentage of how full the parking lot is. These three functionalities will allow drivers to check if there are parking spots prior to driving over, update drivers who are looking for an open lot, and help indicate to drivers where free parking spots are. Figure \ref{fig:cpx} shows what a parking lot row would look like with our system implemented. The light strip at the base of parking spot are green when the spot it empty and off when the spot it filled. There is a billboard LCD at the entrance to the parking lot showing the breakdown of how many empty spots there are per row. On the right side of Figure \ref{fig:cpx} there is a breakdown of the wiring diagram for 3 parking spots. Each parking spot has a sensor and a light strip, which communicate with the microprocessor which updates the website and LCD billboard. According to a study done by the Transportation Research Board \cite{Harding} these systems aid in the experience of parking, making parking a stress free activity so that patrons can effortlessly get to their destination. Students are under enough stress trying to keep up with their project, labs and exams, they do not need added stress of finding a parking space. 

\begin{table}[!h]
  \caption{Parts required for project}
  \label{table:parts_list}
  \begin{center}
  \begin{tabular}{|p{3in}|c|}
  
  \hline
  Part description & Qty\\
  \hline
  \hline
  Adafruit Circuit Playground Express & 1 \\
  \hline
  AAA Battery Holder & 1 \\
  \hline
  USB Cable & 1 \\
  \hline
  Ultrasonic Sensor & 3 \\
  \hline
  Neopixel Strip & 1 \\
  \hline
  LCD & 1 \\
  \hline
  \end{tabular}
  \end{center}
\end{table}
\section{CONCLUSION}
The primary goal of this project is to assist users of a parking lot to save time. We implemented multiple features to be able to achieve this. Our main concept is to install sensors in parking spots that detects the presence of a car. This data is used to calculate the number of parking spots available. We plan on implementing lights at the entry of the parking spot, as shown in the diagram. These will allow the user to see where the empty parking spot is, which allows them to navigate to the spot efficiently. We will display the number of spots left and the row where the empty spot is on an LCD at the entrance of the parking lot. We also plan on displaying this information on a website which will allow users to plan ahead of time.

\newpage
\section{References}
\printbibliography[heading=subbibintoc]
%\bibliographystyle{plain}
%\bibliography{ref.bib}

\end{document}
